\documentclass[11pt]{article}

\usepackage[margin=1in]{geometry}
\usepackage{fancyhdr}

\pagestyle{fancy}

\lhead{Multi-Agent Path Optimization} \rhead{\today}
\renewcommand{\headrulewidth}{0.4pt}
\setlength{\headheight}{13.6pt}

\usepackage{csquotes}
\usepackage{amsmath}

\usepackage{graphicx}
\graphicspath{{figures}}
\usepackage{caption}

\usepackage[linktoc=all]{hyperref}
\usepackage[nameinlink]{cleveref}

%\usepackage{biblatex}
%\addbibresource{references.bib}

\usepackage{lipsum}
\usepackage{xcolor}

\usepackage[toc,page]{appendix}
\usepackage{tocloft}
\renewcommand{\cftsecleader}{\cftdotfill{\cftdotsep}}

\usepackage[numberedsection,symbols,nogroupskip,sort=none]{glossaries-extra}

\glsxtrnewsymbol[description={number of robots}]{N}{\ensuremath{N}}
\glsxtrnewsymbol[description={number of circular obstacles}]{M}{\ensuremath{M}}
\glsxtrnewsymbol[description={initial position of the \(i\)-th robot}]{x_i}{\ensuremath{x_i}}
\glsxtrnewsymbol[description={target position of the \(i\)-th robot}]{y_i}{\ensuremath{y_i}}
\glsxtrnewsymbol[description={path of the \(i\)-th robot}]{p_i}{\ensuremath{p_i}}
\glsxtrnewsymbol[description={radius or repulsive distance of robots}]{r}{\ensuremath{r}}
\glsxtrnewsymbol[description={center of the \(j\)-th circular obstacle}]{c_j}{\ensuremath{c_j}}
\glsxtrnewsymbol[description={radius of circular obstacles}]{R}{\ensuremath{R}}
\glsxtrnewsymbol[description={objective function}]{F}{\ensuremath{F}}
\glsxtrnewsymbol[description={Lagrangian function}]{L}{\ensuremath{L}}
\glsxtrnewsymbol[description={barrier function}]{g}{\ensuremath{g}}

\usepackage{enumitem}
\setlist{  
  listparindent=\parindent,
  parsep=0pt,
}

\usepackage{amsmath}
\usepackage{amsthm}
\usepackage{amssymb}
\usepackage{amsfonts}
\usepackage{mathtools}
\usepackage{multirow}
\usepackage{multicol}
\usepackage{stackrel}
\usepackage{titlesec}
\usepackage{comment}

\begin{comment}
\usepackage[T1]{fontenc}
\usepackage{charter}
\usepackage[charter]{mathdesign}    
\end{comment}
  
% some macros
\newcommand{\set}[1]{\{ #1 \}}
\newcommand{\Bigset}[1]{\Big\{ #1 \Big\}}
\newcommand{\mat}[4]{\begin{bmatrix} #1 & #2 \\ #3 & #4 \end{bmatrix}}
\newcommand{\conj}[1]{\bar{#1}}
%\DeclareMathOperator{\dim}{dim}
\DeclareMathOperator{\im}{im}
\DeclareMathOperator{\nullity}{nullity}
\DeclareMathOperator{\tr}{trace}
\DeclareMathOperator{\rank}{rank}
\DeclareMathOperator{\fcn}{fcn}
\DeclareMathOperator{\Var}{Var}
\DeclareMathOperator{\var}{var}
%\DeclareMathOperator{\sp}{Span}
%\DeclareMathOperator{\det}{det}
\newcommand{\R}{\mathbb{R}}
%\newcommand{\B}{\mathcal{B}}
\newcommand{\C}{\mathbb{C}}
\newcommand{\Z}{\mathbb{Z}}
\newcommand{\N}{\mathbb{N}}
\newcommand{\Q}{\mathbb{Q}}
\renewcommand{\P}{\mathbb{P}} % for probability measure
\renewcommand{\sp}{\text{Span}}
%\renewcommand{\CC}{\textbf{C}}
%\renewcommand{\S}{\mathcal{S}}
\newcommand{\M}{\mathbb{M}}
\newcommand{\al}{\alpha}
\newcommand{\be}{\beta}
%\newcommand{\p}{\varphi}
\newcommand{\ga}{\gamma}
\newcommand{\la}{\lambda}
\newcommand{\eps}{\varepsilon}
\renewcommand{\th}{\theta}
\newcommand{\sse}{\subseteq}
\newcommand{\dd}{\, \mathrm{d}}
\renewcommand{\d}{\mathrm{d}}
\newcommand{\dual}{^*}
\newcommand{\orth}{^{\perp}}
%\renewcommand{\c}{^{\text{c}}}
\newcommand{\lV}{\lVert}
\newcommand{\rV}{\rVert}
\newcommand{\V}{\Vert}
\newcommand{\lv}{\lvert}
\newcommand{\rv}{\rvert}
\renewcommand{\v}{\vert}
\newcommand{\lan}{\left\langle}
\newcommand{\ran}{\right\rangle}
\newcommand{\dir}{\oplus}
\newcommand{\ang}[1]{\left\langle #1 \right\rangle}
\newcommand{\abs}[1]{\left\lvert #1 \right\rvert}
\newcommand{\norm}[1]{\left\lVert #1 \right\rVert}
\newcommand{\adj}{^*}
\newcommand{\blankang}{\ang{\phantom{v},\phantom{v}}}
\renewcommand{\i}{\sqrt{-1}}
\newcommand{\ds}{\displaystyle}
\newcommand{\dsum}{\displaystyle\sum}
\newcommand{\dprod}{\displaystyle\prod}
\DeclareMathOperator{\cis}{cis}
\newcommand{\an}{^0}
%\DeclareMathOperator{\inf}{inf}
%\DeclareMathOperator{\sup}{sup}
\renewcommand{\sp}{\,\,}
\newcommand{\ex}{\exists \,}
\newcommand{\fa}{\forall}
\renewcommand{\l}{\ell}
\newcommand{\overbar}[1]{\mkern 1.5mu\overline{\mkern-1.5mu#1\mkern-1.5mu}\mkern 1.5mu} % custom overline that is slightly smaller
\newcommand{\cl}[1]{\overbar{#1}}
\newcommand{\seq}[1]{\set{#1}_{n\geq 1}}
\newcommand{\dcup}{\displaystyle\bigcup}
\newcommand{\dcap}{\displaystyle\bigcap}
\newcommand{\dliminf}{\displaystyle\liminf}
\newcommand{\dlimsup}{\displaystyle\limsup}
\newcommand{\dlim}{\displaystyle\lim}
\newcommand{\ndlim}{\displaystyle\lim_{n \to \infty}}
\newcommand{\mdlim}{\displaystyle\lim_{m \to \infty}}
\newcommand{\Ndlim}{\displaystyle\lim_{N \to \infty}}
\newcommand{\dint}{\displaystyle\int}
\renewcommand{\qed}{\hfill $\square$}
\newcommand{\evan}{$\phantom{evan}$ \qed}
\newcommand{\kotha}{\dsum_{n=0}^{\infty}}
\newcommand{\kothaa}{\dsum_{n=1}^{\infty}}
\DeclareMathOperator{\sgn}{sgn}
\newcommand{\minus}{\hspace{1pt}\texttt{\char`\\}\hspace{1pt}}
\newcommand{\ba}{\mathbf{a}}
\newcommand{\bb}{\mathbf{b}}

\usepackage{mathrsfs}

\newcommand{\Y}{\mathscr{Y}} % C^1 function vector space
\newcommand{\D}{\mathscr{D}} % domain of functions

% for 2-d vectors
\newcommand{\x}{\mathbf{x}}
\newcommand{\y}{\mathbf{y}}
\newcommand{\z}{\mathbf{z}}
\renewcommand{\c}{\mathbf{c}}
\newcommand{\p}{\mathbf{p}}

\usepackage[symbol]{footmisc}

\begin{document}

\begin{titlepage}

	\centering
	\null
	\vspace{\stretch{1}}
	
	{\huge \textbf{Multi-Agent Path Optimization\\using Calculus of Variations}\par}
	\vspace{5mm}
	{\Large \large Course Project\par
	MATH 146 --- Methods of Applied Mathematics\par}
	\vspace{5mm}
	{\large \textbf{Author:} Robert Lee and Vedant Yogishwar\par
	\textbf{Instructor:} Dr. Shiba Biswal\par}
	
	\vspace{\stretch{2}}
	
	{\large \today\par}

\end{titlepage}

\clearpage

\tableofcontents

\clearpage

\begin{abstract}
\lipsum[1-2]
\end{abstract}

\clearpage

\printunsrtglossary[type=symbols,style=long,title={List of Symbols}]

\section{Introduction}

\section{Theory}

\subsection{Problem Definition}

\textcolor{red}{Add description of problem here.}

Let \(p_i \in C^1([0,1])^2\) be the vector valued function corresponding to the path of the \(i\)-th robot. Then we have that \(p_i(0) = x_i\) and \(p_i(1) = y_i\). The length of the path is defined as
\begin{equation}
	l_i = \int_0^1 \norm{p_i'(t)} \,dt.
\end{equation}

\subsection{Objective Function}

We wish to minimize the total length of all robot paths. Define \(F\) as the total path length
\begin{equation}
	F(P) = \sum_{i=1}^{N} l_i = \int_0^1 \sum_{i=1}^{N} \norm{p_i'(t)} \,dt,
\end{equation}
where \(P = [p_1, \ldots, p_N]^\intercal\) is the path ensemble.

We will modify the objective function in order to represent the no-collision and obstacle constraints. As an example, consider the obstacle constraint. In order to check whether or not a robot collides with an obstacle, a function \(g\) can be defined such that it evaluates to infinity if the distance between the robot and the center of the obstacle is less than \(R\) and evaluates to 0 when not. However, \(g\) would be discontinuous and forbid the use of calculus of variations techniques.

A barrier function \(g_{d,\mu}\) is a continuous function that goes to infinity as it approaches a ``barrier'' value \(d\). A parameter \(\mu\) can be varied in order for \(g\) to approach the discontinuous form. In this case, we wish to devise a function that goes to infinity when approached from above to act as a repulsor. In this project, we will consider two types of barrier functions
\begin{enumerate}
	\item Log function:
	\begin{equation}
		g_{d,\mu}(x) = \frac{-1}{\mu} \log(x-d),
		\quad
		g'_{d,\mu}(x) = \frac{-1}{\mu(x-d)}.
	\end{equation}
	\item Inverse function:
	\begin{equation}
		g_{d,\mu}(x) = \frac{1}{\mu(x-d)},
		\quad
		g'_{d,\mu}(x) = \frac{-1}{\mu(x-d)^2}.
	\end{equation}
\end{enumerate}

Using this barrier function we can rewrite \(F\) as
\begin{equation}
	F(P) = \int_0^1 \left[ \sum_{i=1}^{N} \norm{p'_i(t)} + \sum_{i,j = 1; i \neq j}^{N} g_{r,\mu_1}\left( \norm{p_i(t) - p_j(t)} \right) + \sum_{i=1}^{N} \sum_{j=1}^{M} g_{R,\mu_2}\left( \norm{p_i(t) - c_j} \right) \right] \,dt,
\end{equation}
and the optimization problem becomes an unconstrained optimization problem.

\subsection{First-Order Necessary Conditions}

To derive first-order necessary condition, we first identify the Lagrangian \(L\),
\begin{equation}
	L(Y,Z) = \sum_{i=1}^{N} \norm{z_i} + \sum_{i,j = 1; i \neq j}^{N} g_{r,\mu_1}\left( \norm{y_i - y_j} \right) + \sum_{i=1}^{N} \sum_{j=1}^{M} g_{R,\mu_2}\left( \norm{y_i - c_j} \right).
\end{equation}

Applying the Euler--Lagrange equation,
\begin{align}
	&L_{p'_{i,x}}(t) = \frac{p'_{i,x}(t)}{\norm{p'_i(t)}}
	\quad\implies\quad
	\frac{d}{dt}\left[ L_{p'_{i,x}}(t) \right] = \frac{p''_{i,x}(t)}{\norm{p'_i(t)}} - \frac{p'_{i,x}(t) \big( p'_i(t) \cdot p''_i(t) \big)}{\norm{p'_i(t)}^3}
	\label{eq:EL1} \\	
	&\begin{aligned}
	L_{p_{i,x}}(t) &= \sum_{i,j=1; i \neq j}^{N} g'_{r,\mu_1}\left( \norm{p_i(t) - p_j(t)} \right) \frac{p_{i,x}(t) - p_{j,x}(t)}{\norm{p_{i,x}(t) - p_{j,x}(t)}} \\
	&\qquad\quad+ \sum_{j=1}^{M} g'_{R,\mu_2}\left( \norm{p_i(t) - p_j(t)} \right) \frac{p_{i,x}(t) - c_{j,x}}{\norm{p_i(t) - c_j}}.
	\end{aligned}
	\label{eq:EL2}
\end{align}

Setting \Cref{eq:EL1,eq:EL2} equal to each other, the differential equation formed will be used to find the optimal path ensemble.

\section{Cases}

\section{Results}

\section{Discussion}

\section{Conclusion}

\section{Acknowledgements}



%\nocite{*}
%\printbibliography[heading={bibnumbered}]

\begin{appendices}

\section{MATLAB Code}

\end{appendices}

\end{document}